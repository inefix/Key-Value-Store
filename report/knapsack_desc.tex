%#####################################################################
% Author: Jonas Epper, Ryan Siow & Andrea Rar
% Created: 23.05.2018
% Modified: 26.05.2018
% Description: description projet
%#####################################################################
\documentclass[12pt,a4paper]{article}

\usepackage[head=14pt,left=1in,right=1in, bottom=1in]{geometry} % to modify geometry
\usepackage[utf8]{inputenc}                           % to allow utf-8 encoding input e.g. ö é ä etc.
\usepackage[french]{babel}			                  % other options: frenchb,german
\usepackage[pdftex]{graphicx}                         % to include graphics
\usepackage{amsmath}                                  % for maths functions
\usepackage{amsfonts}                                 % for maths fonts
\usepackage{amssymb}                                  % for maths symbols
\usepackage[hidelinks]{hyperref}                      % to allow urls and inner references
\usepackage{titling}                                  % for fancy titles
\usepackage{caption}                                  % for fancy captions
\usepackage{listings}                                 % to include listings
\usepackage{color}                                    % for custom colors

\usepackage{blindtext}                                % for placeholder text
\usepackage{float}									  % for fixing figure


% setup page
%\sloppy
%\pagenumbering{roman}
%\pagestyle{plain}
%\pagenumbering{arabic}
%\pagestyle{headings}

% header and footer
\usepackage{lastpage}
\usepackage{fancyhdr}
\pagestyle{fancy}
\fancyfoot[R]{\thepage/\pageref{LastPage}}
\fancyfoot[C]{}
\fancyfoot[L]{\footnotesize\emailurl{\myemail}}
\fancyhead[L]{\thetitle}
\fancyhead[R]{\theauthor}

% remove additional space between tables and their captions
%\captionsetup[table]{skip=0pt}

\graphicspath{{figures/}}


%%%%%%%%%%%%%%%%%%%%
% Listings

\definecolor{codegreen}{rgb}{0,0.6,0}
\definecolor{codegray}{rgb}{0.5,0.5,0.5}
\definecolor{codepurple}{rgb}{0.58,0,0.82}
\definecolor{backcolour}{rgb}{0.99,0.99,0.97}

\lstdefinestyle{customc}{
  belowcaptionskip=1\baselineskip,
  breaklines=true,
  frame=single, 
  tabsize=4, 
  numbers=left,
  numberstyle=\tiny\color{codegray},
  stepnumber=1,
  xleftmargin=\parindent,
  language=Erlang,
  showstringspaces=false,
  basicstyle=\fontsize{9}{11}\selectfont\ttfamily,
  %basicstyle=\tiny,
  backgroundcolor=\color{backcolour},   
  commentstyle=\color{codegreen},
  keywordstyle=\color{codepurple},
  stringstyle=\color{blue},
  identifierstyle=\color{black},
  extendedchars=true,
  inputencoding=utf8,
  literate={á}{{\'a}}1 {à}{{\'a}}1 {ã}{{\~a}}1 {é}{{\'e}}1 {è}{{\'e}}1 {ê}{{\'e}}1 {ç}{{\'c}}1 
}

\lstset{escapechar=@,style=customc} % set default language

%\lstset{language=Erlang} 

%%%%%%%%%%%%%%%%%%%%
% TODO box

\newcommand{\todo}[1]{\par \noindent
\begin{minipage}[c]{0.95 \textwidth}
\textit{#1} \end{minipage}\par}

\newcommand{\emailurl}[1]{\href{mailto:#1}{#1}}
\newcommand{\email}[1]{\gdef\myemail{#1}}
\newcommand{\address}[1]{\gdef\myaddress{#1}}
%\newcommand{\keywords}[1]{\begin{center}{\bfseries Keywords:} #1\end{center}}
\newcommand{\keywords}[1]{\vskip \baselineskip \noindent{\bfseries Keywords:} #1 \vskip \baselineskip \par}


\renewcommand\maketitle{
\begin{center}%
    {\LARGE \thetitle \vspace \baselineskip \par}%
    {\large \theauthor \vspace \baselineskip \par}%
\myaddress
\emailurl{\myemail}
\end{center}
}

%################### Preamble Stop #####################################

% title setup
\title{\vspace{-10em}Key-value Store: Project 2018}
\date{\today}
\author{Andrea \textsc{Rar}, Ryan \textsc{Siow}, Jonas \textsc{Epper}}
\address{IN.4022 Operating System Course, University of Fribourg \\}
\email{andrea.rar@unifr.ch, ryan.siow@unifr.ch, jonas.epper@unifr.ch}

\begin{document}
%################### Report Start 
\maketitle
\thispagestyle{empty}

\renewcommand{\contentsname}{\large Table des matières}
\renewcommand{\baselinestretch}{0.75}\normalsize
\tableofcontents % se fait automatiquement

\section{Introduction}
Tous. rester TRES COURT et CONCIS

\section{Description du problème}
Tous. décrire le problème en gros, on ira en profondeur plus tard

Plusieurs problèmes devaient être traités durant ce projet. Nous avons aussi dû prendre des décisions d'implémentation qui rendait notre solution unique et si possible proche de l'efficacité optimale.\\
La première étape était de créer une structure permettant de recevoir des paires de clés et de valeurs. Cette structure devait être dynamique, c'est-à-dire grandir à mesure que l'on ajoutait des paires afin d'utiliser le moins de mémoire possible. Cette structure devait supporter des actions telles que ajouter, modifier, lire ou supprimer des paires.\\
De plus, il fallait gérer une communication entre le serveur et ses clients. le serveur devait être transparent pour les clients et ceux-ci devait juste pouvoir envoyer des requêtes simples et le serveur leurs retournait les résultats des requêtes. \\
Finalement, les accès à la structure de données devait être sécurisés afin qu'aucun conflit d'occure. Il ne faudrait par exemple pas que deux clients modifie en même temps la même pair, ceci pouvant générer des comportements inattendus et imprévisibles. 
\section{Solutions et décisions prises}
Dans cette partie, nous allons décrire les solutions que nous avons élaboré pour résoudre chaque problème mentionné auparavant. 
\subsection{KV store: array dynamique de struct}
Ryan
nombre d'entrée max? taille variable? 
\subsection{communication server - clients}
nombres de client max
Andrea : renvoie des messages aux client
Jonas tout le reste
\subsection{lire,écrire et modifier des valeurs}
Tous
Regex,
\subsection{Accès sécurisé des lecteurs/rédacteurs}
Andrea
qui a la priorité? read/write etc
nombre de clients max? reader/writer nombres
citer la page web pour le problème des lecteurs/rédacteurs
\subsection{scripts de tests}
Jonas

italique: \textit{exemple italique}

\section{apprentissage}
ce que nous avons appris
Tous

\section{Conclusion}


% Example image
%\begin{figure}[ht]
%	\centering
%  %\includegraphics[height=200px]{graphs/TEDAClock.png} % chemin jusqu'à l'image
%\caption{TEDA Clock}
%	\label{fig:fig4}
%\end{figure}

%------------------- Bibliography Start -------------------------------
\begin{thebibliography}{20}

% book ressources example
\bibitem{article} % cited with '\cite{book}'
????			% author (Firstname Lastname, Firstname2 Lastname2, ...)
\textit{titre de l'article, ref}.	% title (italics)
Read the ....										% editor, date

% web source example
\bibitem{dynamique}
???			% author (sometimes not available)
\textit{faire un tableau dynamique en C}.					    % title (italics)
\url{http://www....},	% url
Last visited: ....								% date
\end{thebibliography}
%------------------- Bibliography End -------------------------------


%################### Report End #####################################


%################### Appendix Start #####################################
%%%%%%%%%%%%      l'APPENDIX SE MET N'IMPORTE OU... 
\appendix 
%\renewcommand{\thesection}{Appendix \Alph{section}}
\renewcommand{\thesubsection}{\Alph{section}.\arabic{subsection}}

\section{Manuel d'utilisateur} \label{app:annexe}
Le fichier README vous explique de manière détaillée comment compiler et exécuter le programme.


%################### Appendix End #####################################
\end{document}
