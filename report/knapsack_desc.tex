%#####################################################################
% Author: Loïc Rosset & Jonas Epper
% Created: 19.11.2017
% Modified: 1.12.2017
% Description: description projet
%#####################################################################
\documentclass[12pt,a4paper]{article}

\usepackage[head=14pt,left=1in,right=1in, bottom=1in]{geometry} % to modify geometry
\usepackage[utf8]{inputenc}                           % to allow utf-8 encoding input e.g. ö é ä etc.
\usepackage[french]{babel}			                  % other options: frenchb,german
\usepackage[pdftex]{graphicx}                         % to include graphics
\usepackage{amsmath}                                  % for maths functions
\usepackage{amsfonts}                                 % for maths fonts
\usepackage{amssymb}                                  % for maths symbols
\usepackage[hidelinks]{hyperref}                      % to allow urls and inner references
\usepackage{titling}                                  % for fancy titles
\usepackage{caption}                                  % for fancy captions
\usepackage{listings}                                 % to include listings
\usepackage{color}                                    % for custom colors

\usepackage{blindtext}                                % for placeholder text
\usepackage{float}									  % for fixing figure
%################### Preamble Start 
%%%%%%%%%%%%%%%%%%%%
% move title to top of page
\setlength{\droptitle}{-10em}

%%%%%%%%%%%%%%%%%%%%
% setup page
%\sloppy
%\pagenumbering{roman}
%\pagestyle{plain}
%\pagenumbering{arabic}
%\pagestyle{headings}

% header and footer
\usepackage{lastpage}
\usepackage{fancyhdr}
\pagestyle{fancy}
\fancyfoot[R]{\thepage/\pageref{LastPage}}
\fancyfoot[C]{}
\fancyfoot[L]{\footnotesize\emailurl{\myemail}}
\fancyhead[L]{\thetitle}
\fancyhead[R]{\theauthor}

% remove additional space between tables and their captions
%\captionsetup[table]{skip=0pt}

\graphicspath{{figures/}}


%%%%%%%%%%%%%%%%%%%%
% Listings

\definecolor{codegreen}{rgb}{0,0.6,0}
\definecolor{codegray}{rgb}{0.5,0.5,0.5}
\definecolor{codepurple}{rgb}{0.58,0,0.82}
\definecolor{backcolour}{rgb}{0.99,0.99,0.97}

\lstdefinestyle{customc}{
  belowcaptionskip=1\baselineskip,
  breaklines=true,
  frame=single, 
  tabsize=4, 
  numbers=left,
  numberstyle=\tiny\color{codegray},
  stepnumber=1,
  xleftmargin=\parindent,
  language=Erlang,
  showstringspaces=false,
  basicstyle=\fontsize{9}{11}\selectfont\ttfamily,
  %basicstyle=\tiny,
  backgroundcolor=\color{backcolour},   
  commentstyle=\color{codegreen},
  keywordstyle=\color{codepurple},
  stringstyle=\color{blue},
  identifierstyle=\color{black},
  extendedchars=true,
  inputencoding=utf8,
  literate={á}{{\'a}}1 {à}{{\'a}}1 {ã}{{\~a}}1 {é}{{\'e}}1 {è}{{\'e}}1 {ê}{{\'e}}1 {ç}{{\'c}}1 
}

\lstset{escapechar=@,style=customc} % set default language

%\lstset{language=Erlang} 

%%%%%%%%%%%%%%%%%%%%
% TODO box

\newcommand{\todo}[1]{\par \noindent
\begin{minipage}[c]{0.95 \textwidth}
\textit{#1} \end{minipage}\par}

\newcommand{\emailurl}[1]{\href{mailto:#1}{#1}}
\newcommand{\email}[1]{\gdef\myemail{#1}}
\newcommand{\address}[1]{\gdef\myaddress{#1}}
%\newcommand{\keywords}[1]{\begin{center}{\bfseries Keywords:} #1\end{center}}
\newcommand{\keywords}[1]{\vskip \baselineskip \noindent{\bfseries Keywords:} #1 \vskip \baselineskip \par}

\renewcommand\maketitle{
\begin{center}%
    {\huge \thetitle \vskip \baselineskip \par}%
    {\large \theauthor \vskip \baselineskip \par}%
\myaddress
\emailurl{\myemail}
\end{center}
}

%################### Preamble Stop #####################################

% title setup
\title{Key-value Store: mini-Project 2018}
\date{\today}
\author{Andrea \textsc{Rar}, Ryan \textsc{Siow}, Jonas \textsc{Epper}}
\address{IN.4022 Operating system\\
University of Fribourg \\}
\email{andrea.rar@unifr.ch, ryan.siow@unifr.ch, jonas.epper@unifr.ch}

\begin{document}
%################### Report Start 
\maketitle
\thispagestyle{empty}

\begin{abstract}
\noindent Résumé

\keywords{Key-valuestore, threads,...}

\end{abstract}

\newpage
\renewcommand{\contentsname}{Table des matières}
\tableofcontents
\newpage

\section{Introduction}
Intro...

\section{titre}

italique: \textit{exemple italique}
\subsection{soustitre}


\begin{figure}[ht]
	\centering
  %\includegraphics[height=200px]{graphs/TEDAClock.png} % chemin jusqu'à l'image
\caption{TEDA Clock}
	\label{fig:fig4}
\end{figure}


\section{Conclusion}
...

%------------------- Bibliography Start -------------------------------
\begin{thebibliography}{20}

\bibitem{article} % cited with '\cite{book}'
Ananth Murthy, Chandan Yeshwanth, Shrisha Rao.									% author (Firstname Lastname, Firstname2 Lastname2, ...)
\textit{Distributed Approximation Algorithms for the Multiple Knapsack Problem}.	% title (italics)
2 Février 2017.										% editor, date

\bibitem{Karp} % cited with '\cite{book}'
Richard M. Karp.									% author (Firstname Lastname, Firstname2 Lastname2, ...)
\textit{Reducibility Among Combinatorial Problems}.	% title (italics)
R.E. Miller et J. W. Thatcher, 1972.										% editor, date

\bibitem{Cook} % cited with '\cite{book}'
Stephen Cook.									% author (Firstname Lastname, Firstname2 Lastname2, ...)
\textit{The Complexity of Theorem-Proving Procedures}.	% title (italics)
Conference Record of Third Annual ACM Symposium on Theory of Computing (STOC), 1971.										% editor, date

\bibitem{Crypto} % cited with '\cite{book}'
Ralph C. Merkle, Martin E. Hellman.									% author (Firstname Lastname, Firstname2 Lastname2, ...)
\textit{Hiding information and signatures in trapdoor knapsacks}.	% title (italics)
IEEE Transaction on Information Theory, 1978.

% web source
\bibitem{PNP}
Thomas Messias.								% author (sometimes not available)
\textit{P = NP ou P =/= NP, le problème de maths à un million de dollars}.					    % title (italics)
\url{http://www.slate.fr/story/109569/probleme-million-dollars},			% url
Last visited: 17.12.2017.									% date
\end{thebibliography}
%------------------- Bibliography End -------------------------------


%################### Report End #####################################


%################### Appendix Start #####################################
%%%%%%%%%%%%      l'APPENDIX SE MET N'IMPORTE OU... 
\appendix 
%\renewcommand{\thesection}{Appendix \Alph{section}}
\renewcommand{\thesubsection}{\Alph{section}.\arabic{subsection}}

\clearpage	

\section{Code source} \label{app:sourceCode}

\subsection{Phase d'initialisation}
Exemple sous titre

\begin{figure}[H]
%\lstinputlisting[firstline=17, lastline=50]{../code/greedy.erl}
\caption{Phase d'initialisation.}
\label{fig:initalisation}
\end{figure}

\section{User Manual}
Pour un guide d'utilisation des codes, voir le fichier ....., en annexe.


%\begin{figure}
%\begin{lstlisting}
%// constrain speed to +/- MAX_SPEED
%double bounded_speed(double speed) { 
%  if (speed > MAX_SPEED) return MAX_SPEED;
%  else if (speed < -MAX_SPEED) return -MAX_SPEED;
%  else return speed;
%}
%\end{lstlisting}
%\caption{Listing within \LaTeX.}
%\label{fig:listing2}
%\end{figure}



%################### Appendix End #####################################
\end{document}
